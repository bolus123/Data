This handout will introduce a basic method of simulation. This pdf file and the R program are provided on Blackboard.
%This pdf file and the R program are provided on my Github: \par\href{url}{https://github.com/bolus123/R-handout/tree/master/DescriptiveStatistics}
\begin{enumerate}
	\item Prerequisite: Probability Integral Transform (PIT)
	\par Suppose that a random variable X has a continuous distribution for which the cumulative distribution function (CDF) is $F_X$. Then the random variable $Y$ defined as
	\begin{equation*}
		Y = F_X(X) \sim Uniform(0,1)
	\end{equation*}
	\begin{itemize}
		\item R code
		\begin{verbatim}
			# Set a seed to make this process repeatable
			set.seed(12345) 
		
			# Set the number of simulations
			n <- 100000
			
			# Simulate data from a standard normal distribution
			X <- rnorm(n)
			
			# Find out the coresponding quantiles
			Y <- pnorm(X)
			
			# Use histogram to graph the distribution
			hist(Y, freq = F, ylim = c(0, 1.5))
		\end{verbatim}
	\end{itemize}
	\item Simulate data from a standard normal distribution
		\begin{enumerate}
			\item Direct simulation
				\begin{itemize}
					\item R code
					\begin{verbatim}
					# Set a seed to make this process repeatable
					set.seed(12345) 
					
					# Set the number of simulations
					n <- 100000
					
					# Simulate data from a standard normal distribution
					X <- rnorm(n)
					
					# Use histogram to graph the distribution
					hist(X, freq = F)
					\end{verbatim}
				\end{itemize}
				
			\item Indirect simulation by PIT
				\begin{itemize}
					\item R code
					\begin{verbatim}
					# Set a seed to make this process repeatable
					set.seed(12345) 
					
					# Set the number of simulations
					n <- 100000
					
					# Simulate data from Uniform(0,1)
					Y <- runif(n)
					
					# By PIT, get a sample from a standard normal distribution
					X <- qnorm(Y)
					
					# Use histogram to graph the distribution
					hist(X, freq = F)
					\end{verbatim}
				\end{itemize}
				
		\end{enumerate}
	\item Simulate data from a contaminated normal distribution. Suppose we have a contaminated normal distribution as section 3.4.1 in the 7th edition. Because $W = ZI_{1-\epsilon} + \sigma_c Z (1 - I_{1 - \epsilon})$, we have this definition
		\begin{equation*}
			W = XI_{1-\epsilon} + Y (1 - I_{1 - \epsilon})
		\end{equation*}
	where $X \sim N(0,1)$, $Y \sim N(0, \sigma_c^2)$ and $I \sim Bernoulli(1 - eps)$
		\begin{itemize}
			\item R code
			\begin{verbatim}
				# Set a seed to make this process repeatable
				set.seed(12345) 
				
				# Set the number of simulations
				n <- 100000
				
				# Build a function to simulate a contaminated normal distribution
				rctnorm <- function(n, eps = 0.5, mu = c(0, 0), 
				sigma = c(1, 1)) {
				
				# Simulate I
				I <- rbinom(n, 1, 1 - eps)
				# Simulate X
				X <- rnorm(n, mu[1], sigma[1])
				# Simulate Y
				Y <- rnorm(n, mu[2], sigma[2])
				
				# Simulate W
				W <- rep(NA, n)
				for (i in 1:n){
				W[i] <- ifelse(I[i] == 1, X[i], Y[i])
				}
				
				W
				
				}
				
				# the p.d.f of contaminated normal distribution
				dctnorm <- function(x, eps = 0.5, mu = c(0, 0), 
				sigma = c(1, 1)){
				
				dnorm(x, mu[1], sigma[1]) * (1 - eps) + 
				dnorm(x, mu[2], sigma[2]) * eps
				
				}
				
				x <- seq(-100, 100, 0.01)
				
				par(mfrow = c(1, 1))
				
				# 3.4.26 part b, eps = 0.15, mu1 = mu2 = 0, 
				# sigma1 = 1, sigma2 = sigma.c = 10
				X1 <- rctnorm(n, 0.15, mu = c(0, 0), sigma = c(1, 10))
				# Use histogram to graph the distribution
				hist(X1, freq = F, main = 'eps = 0.15 and sigma.c = 10'
				, ylim = c(0, 0.45))
				points(x, dnorm(x, 0, 1), type = 'l', col = 'red')
				points(x, dnorm(x, 0, 10), type = 'l', col = 'blue')
				points(x, dctnorm(x, 0.15, c(0, 0), c(1, 10)), type = 'l'
				, col = 'black')
				
				# 3.4.26 part c, eps = 0.15, mu1 = mu2 = 0, 
				# sigma1 = 1, sigma2 = sigma.c = 20
				X2 <- rctnorm(n, 0.15, mu = c(0, 0), sigma = c(1, 20))
				# Use histogram to graph the distribution
				hist(X2, freq = F, main = 'eps = 0.15 and sigma.c = 20'
				, ylim = c(0, 0.45))
				points(x, dnorm(x, 0, 1), type = 'l', col = 'red')
				points(x, dnorm(x, 0, 20), type = 'l', col = 'blue')
				points(x, dctnorm(x, 0.15, c(0, 0), c(1, 20)), type = 'l'
				, col = 'black')
				
				# 3.4.26 part d, eps = 0.25, mu1 = mu2 = 0, 
				# sigma1 = 1, sigma2 = sigma.c = 20
				X3 <- rctnorm(n, 0.25, mu = c(0, 0), sigma = c(1, 20))
				# Use histogram to graph the distribution
				hist(X3, freq = F, main = 'eps = 0.25 and sigma.c = 20'
				, ylim = c(0, 0.45))
				points(x, dnorm(x, 0, 1), type = 'l', col = 'red')
				points(x, dnorm(x, 0, 20), type = 'l', col = 'blue')
				points(x, dctnorm(x, 0.25, c(0, 0), c(1, 20)), type = 'l'
				, col = 'black')
				
				# An example when we have 2 normal distribution 
				# with different locations. eps = 0.5, mu1 = 0, 
				# mu2 = 5, sigma1 = sigma2 = 1
				X4 <- rctnorm(n, 0.5, mu = c(0, 5), sigma = c(1, 1)) 
				hist(X4, freq = F, main = 'eps = 0.5 with different locations 
				and same sigmas', ylim = c(0, 0.45))
				points(x, dnorm(x, 0, 1), type = 'l', col = 'red')
				points(x, dnorm(x, 5, 1), type = 'l', col = 'blue')
				points(x, dctnorm(x, 0.5, c(0, 5), c(1, 1)), type = 'l'
				, col = 'black')
				15
			\end{verbatim}
		\end{itemize}

	\item Practice: Simulate data from a gamma distribution with $\alpha = 2$ and $\beta = 5$. (You will need to use rgamma and help(rgamma) in R) 
\end{enumerate}