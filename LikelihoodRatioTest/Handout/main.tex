\documentclass[12pt]{article}
\usepackage{graphicx,subfigure,amsthm,amsmath,latexsym,amssymb,times,asa}
\usepackage{float,epsfig,multirow,rotating,times,verbatim,wrapfig,color}
\RequirePackage{natbib}
\usepackage{hyperref}
\hypersetup{
	colorlinks=true,
	linkcolor=blue,
	filecolor=magenta,      
	urlcolor=cyan,
}

\urlstyle{same}
\setlength{\oddsidemargin}{0in}
\setlength{\evensidemargin}{0in}
\setlength{\textwidth}{6.5in}
\setlength{\topmargin}{-0.6in}
\setlength{\textheight}{9in}
\evensidemargin \oddsidemargin
\newtheorem{theorem}{Theorem}[section]
\newtheorem{lemma}[theorem]{Lemma}
\newtheorem{corollary}[theorem]{Corollary}
\newtheorem{result}[theorem]{Result}
\newcommand{\abs}[1]{|#1|}
\newcommand{\norm}[1]{\Vert#1\Vert}
\newcommand{\R}{I\!\!R}
\newcommand{\bell}{\boldsymbol{\ell}}
\newcommand{\bd}{\boldsymbol{d}}
\newcommand{\blambda}{\boldsymbol{\lambda}}
\newcommand{\balpha}{\boldsymbol{\alpha}}
\newcommand{\bpi}{\boldsymbol{\pi}}
\newcommand{\btheta}{\boldsymbol{\theta}}
\newcommand{\bnabla}{\boldsymbol{\nabla}}
\newcommand{\bLambda}{\boldsymbol{\Lambda}}
\newcommand{\bDelta}{\boldsymbol{\Delta}}
\newcommand{\bkappa}{\boldsymbol{\kappa}}
\newcommand{\bbeta}{\boldsymbol{\beta}}
\newcommand{\bdelta}{\boldsymbol{\delta}}
\newcommand{\bphi}{\boldsymbol{\phi}}
\newcommand{\bPhi}{\boldsymbol{\Phi}}
\newcommand{\bPi}{\boldsymbol{\Pi}}
\newcommand{\bChii}{\boldsymbol{\Chi}}
\newcommand{\bXi}{\boldsymbol{\Xi}}
\newcommand{\bxi}{\boldsymbol{\xi}}
\newcommand{\btau}{\boldsymbol{\tau}}
\newcommand{\bGamma}{\boldsymbol{\Gamma}}
\newcommand{\bgamma}{\boldsymbol{\gamma}}
\newcommand{\bSigma}{\boldsymbol{\Sigma}}
\newcommand{\bPsi}{\boldsymbol{\Psi}}
\newcommand{\bpsi}{\boldsymbol{\psi}}
\newcommand{\bepsilon}{\boldsymbol{\epsilon}}
\newcommand{\bUpsilon}{\boldsymbol{\Upsilon}}
\newcommand{\bmu}{\boldsymbol{\mu}}
\newcommand{\bzeta}{\boldsymbol{\zeta}}
\newcommand{\bvarrho}{\boldsymbol{\varrho}}
\newcommand{\bvarphi}{\boldsymbol{\varphi}}
\newcommand{\bTheta}{\boldsymbol{\Theta}}
\newcommand{\bvartheta}{\boldsymbol{\vartheta}}
\newcommand{\bOmega}{\boldsymbol{\Omega}}
\newcommand{\bomega}{\boldsymbol{\omega}}
\newcommand{\prm}{^{\prime}}
\newcommand{\bB}{\boldsymbol{B}}
\newcommand{\bC}{\boldsymbol{C}}
\newcommand{\bc}{\boldsymbol{c}}
\newcommand{\bD}{\boldsymbol{D}}
\newcommand{\bff}{\boldsymbol{f}}
\newcommand{\bV}{\boldsymbol{V}}
\newcommand{\bG}{\boldsymbol{G}}
\newcommand{\bH}{\boldsymbol{H}}
\newcommand{\bmm}{\boldsymbol{m}}
\newcommand{\bM}{\boldsymbol{M}}
\newcommand{\bi}{\boldsymbol{i}}
\newcommand{\bI}{\boldsymbol{I}}
\newcommand{\bJ}{\boldsymbol{J}}
\newcommand{\bK}{\boldsymbol{K}}
\newcommand{\bk}{\boldsymbol{k}}
\newcommand{\bA}{\boldsymbol{A}}
\newcommand{\bS}{\boldsymbol{S}}
\newcommand{\bt}{\boldsymbol{t}}
\newcommand{\bT}{\boldsymbol{T}}
\newcommand{\bE}{\boldsymbol{E}}
\newcommand{\be}{\boldsymbol{e}}
\newcommand{\bp}{\boldsymbol{p}}
\newcommand{\bP}{\boldsymbol{P}}
\newcommand{\bq}{\boldsymbol{q}}
\newcommand{\bQ}{\boldsymbol{Q}}
\newcommand{\bR}{\boldsymbol{R}}
\newcommand{\bW}{\boldsymbol{W}}
\newcommand{\bu}{\boldsymbol{u}}
\newcommand{\bv}{\boldsymbol{v}}
\newcommand{\bs}{\boldsymbol{s}}
\newcommand{\bU}{\boldsymbol{U}}
\newcommand{\bx}{\boldsymbol{x}}
\newcommand{\bX}{\boldsymbol{X}}
\newcommand{\by}{\boldsymbol{y}}
\newcommand{\bY}{\boldsymbol{Y}}
\newcommand{\bZ}{\boldsymbol{Z}}
\newcommand{\bz}{\boldsymbol{z}}
\newcommand{\blda}{\boldsymbol{a}}
\newcommand{\bOO}{\boldsymbol{O}}
\newcommand{\bw}{\boldsymbol{w}}
\newcommand{\bee}{\boldsymbol{e}}
\newcommand{\bzero}{\boldsymbol{0}}
\newcommand{\lsb}{\bigl\{}
\newcommand{\rsb}{\bigr\}}
\newcommand{\bj}{{\bf{J}}}
\newcommand{\bone}{\boldsymbol{1}}
\newcommand{\bmV}{\boldsymbol{\mathcal V}}
\newcommand{\bmI}{\boldsymbol{\mathcal I}}
\newcommand{\bmA}{\boldsymbol{\mathcal A}}
\newcommand{\mM}{\mathcal M}
\newcommand{\mD}{\mathcal D}
\newcommand{\mR}{\mathcal R}
\newcommand{\mF}{\mathcal F}
\newcommand{\mK}{\mathcal K}
\newcommand{\mI}{\mathcal I}
\newcommand{\mB}{\mathcal B}
\newcommand{\mL}{\mathcal L}
\newcommand{\mG}{\mathcal G}
\newcommand{\one}{$\phantom{1}$}
\newcommand{\oo}{$\phantom{00}$}
\newcommand{\ooo}{$\phantom{000}$}
\newcommand{\oc}{$\phantom{0,}$}
\newcommand{\ooc}{$\phantom{00,}$}
\newcommand{\oooc}{$\phantom{000,}$}
\newcommand{\Ncand}{N_{\rm cand}}
\newcommand{\tover}{t_{\rm over}}
\newcommand{\tstd}{t_{\rm std,1}}
\newcommand{\ttwostage}{t_{\rm 2S,1}}
\newcommand{\psp}{P(S')}
\newcommand{\pep}{P(E')}
\newcommand{\pacc}{P_{\rm acc}}
\newcommand{\prob}{\mbox{I}\!\mbox{Pr}}
\DeclareMathOperator*{\argmin}{argmin}
\DeclareMathOperator*{\argmax}{argmax}
\hyphenation{Lan-ge-vin Lan-ge-vins}
\newcommand{\apprasym}{
 \mathrel{\ooalign{$\sim$\cr\kern+1.25pt\large $\colon$}}}


\newcommand{\floor}[1]{{\lfloor{#1}\rfloor}}
\begin{document}
\markboth{Author}{Title}
\thispagestyle{empty}
\vspace{-1in}
\title{R Handout: Likelihood Ratio Test}
%\author{{Yuhui Yao}
%}
\date{\vspace{-0.3in}}
\maketitle
\renewcommand\baselinestretch{1.33}
%\renewcommand\baselinestretch{1.4}

%\begin{abstract}
%\input{abstract}

%{\bf Keywords: }
%\end{abstract}

\normalsize

\section{Methodology}
\begin{enumerate}
	\item Box-Cox transformation and regression
		\begin{enumerate}
			\item Motivation
			\par In practice, most of our data do not follow any normal distribution. The first attempt we can use is to transform this data into a normal distribution and then follow the traditional methods based on the normal distribution.
			\item Issue
			\par  This transformation change the physical unit. We need to transform it back to the original unit. 
			\item We still learn the "break-into-commercial" sample from the data, \textit{Crime in Vancouver}.
			Suppose the original sample is $X$ and the transformed sample is $Y$. The transformation used here is the squared-root transformation. We have this assumption
			\begin{equation*}
			Y_i = \sqrt{X_i}
			\end{equation*}
			where we assume $Y_i$ follows a normal distribution with mean $\mu_i$ and variance $\sigma^2$. Also, let $\mu_i = at_i^2 + b t_i + c$. The matrix form of the normal distribution
			\begin{equation*}
			f_{\underline{Y}}(\underline{y}) = (2\pi)^{-n/2}(|\underline{\Sigma}|)^{-1/2}e^{-\frac{1}{2} (\underline{y} - \underline{\mu})^{T} \underline{\Sigma}^{-1} (\underline{y} - \underline{\mu}) }
			\end{equation*}
			The log-likelihood function
			\begin{equation*}
			lnL(\underline{\mu}, \underline{\Sigma}, \theta; \underline{y}) = -\frac{n}{2}ln(2\pi) - \frac{1}{2}ln( |\underline{\Sigma}| ) -\frac{1}{2} (\underline{y} - \underline{\mu})^{T} \underline{\Sigma}^{-1} (\underline{y} - \underline{\mu})
			\end{equation*}
			where the vector $\underline{y} = \{ y_1, y_2, ..., y_n \}^T$, $y_i = \sqrt{x_i}$, the mean vector $\underline{\mu} = \{ \mu_1, \mu_2, ..., \mu_ n \}^T$ and the n by n covariance matrix $\underline{\Sigma}$ with diagonal elements $\sigma^2$ and off-diagonal elements $0$. And this is the target we need to maximize. After estimating the parameters, we need to transform it back to our original unit.
			\begin{equation*}
				\begin{split}
					Y_i \sim N(\mu_i, \sigma^2) \Rightarrow \sqrt{X_i} \sim N(\mu_i, \sigma^2)
				\end{split}
			\end{equation*}
			\begin{equation*}
				\Rightarrow \frac{\sqrt{X_i}}{\sigma} \sim N(\mu_i, 1) \Rightarrow \frac{X_i}{\sigma^2} \sim \chi^2_{1, \mu_i} \underrightarrow{D} N(1 + \mu_i, 2(1 + \mu_i))
			\end{equation*}
			where $\chi^2_{1, \mu_i}$ is the noncentral chisquare distribution with 1 degree of freedom and noncentral parameter $\mu_i$.
			\begin{equation*}
				The log-likehood
			\end{equation*}
		\end{enumerate}
	\item Another model based on Poisson.
\end{enumerate}
%\input{research}

\bibliographystyle{plain}
%\bibliography{references}

\begin{thebibliography}{9}
	
	\bibitem{Iris}
	Fisher, R.A. (1936), \textit{The use of multiple measurements in taxonomic problems}, Annual Eugenics, 7, Part II, 179-188.
	
	\bibitem{Wilks}
	Wilks, S. S. (1938)
	\textit{The Large-Sample Distribution of the Likelihood Ratio for Testing Composite Hypotheses},
	The Annals of Mathematical Statistics. \textbf{9}: 60-62.
	
\end{thebibliography}

\end{document}
