This handout will introduce a basic use of R with an example describing a data. This pdf file and the R program are provided on my Github: \par\href{url}{https://github.com/bolus123/R-handout/tree/master/DescriptiveStatistics}
\begin{enumerate}
	\item Load a data. Here, I will only show how to read a CSV file from my Github. You can also load a data from different types of data source, such as Excel, SQL server and etc.. Because I am not going to cover all of them, if you are interested in other methods, please see:
	\par \href{url}{https://cran.r-project.org/doc/manuals/r-devel/R-data.html} 
	\par This CSV file contains a data monitoring a manufacturing process about Automobile Engine Piston Rings \cite{Montgomery09}. Theoretically, each column is identically and independently distributed with each other.
		\begin{itemize}
			\item R code
				\begin{verbatim}
				# Load table from my Github
				add <- 'https://raw.githubusercontent.com/
				bolus123/R-handout/master/DescriptiveStatistics/example.csv'
				data <- as.matrix(read.csv(file = add))
				data
				\end{verbatim}
		\end{itemize}		
	\item Describe this data with graphs
		\begin{itemize}
			\item R code
				\begin{verbatim}
				 	# Graph this data
				 	# histogram for the whole data with 20 breaks
				 	hist(data, breaks = 20) 
				 	
				 	#boxplot for the whole data
				 	boxplot(as.vector(data)) 
				 	
				 	#boxplot for each column
				 	boxplot(data) 
				\end{verbatim}
		\end{itemize}
	\item Describe this data with statistics
		\begin{itemize}
			\item R code
				\begin{verbatim}
					# Basic statistics
					mean(data) #grand mean
					colMeans(data) # means for each column
					rowMeans(data) # means for each row
					
					var(as.vector(data)) # grand variance
					var(data) #covariance matrix
					
					# percentiles including 1%, 5%, 10%, 25%, 
					# 50%, 75%, 90%, 95% and 99%
					quantile(data 
						, c(0.01, 0.05, 0.1, 0.25, 0.5, 0.75, 0.9, 0.95, 0.99)) 
				\end{verbatim}
		\end{itemize}
	\item Fit a model. Suppose we guess this data is following a normal distribution, and then fit a model.
		\begin{itemize}
			\item R code
				\begin{verbatim}
					# fit a model based on the univariate normal distribution 
					# for the whole data
					mu <- mean(data) #grand mean
					sigma <- sqrt(var(as.vector(data))) #standard deviation
					
					# histogram for the whole data with 20 breaks
					hist(data, breaks = 20, freq = FALSE, ylim = c(0, 40))
					curve(dnorm(x, mean = mu, sd = sigma), add = T, col = 'blue')
				\end{verbatim}
		\end{itemize}
	\item Check the normality. We need to verify our normal assumption, because there is no guarantee that we are right. Here, I will show a verification by a Q-Q plot.
		\begin{itemize}
			\item R code
				\begin{verbatim}
					# check the normality (Q-Q plot)
					# we need to have the empirical quantile 
					# and the theoretical quantile based 
					# on the empirical probability
					
					# 1. we need to know the whole sample size
					n <- dim(data)[1] * dim(data)[2]
					
					# 2. sort the data and this is our empirical quantile
					e.q <- sort(data)
					
					# 3. calculate the Empirical cdf
					e.p <- 1:n / n
					
					# 4. find out the theoretical quantile
					t.q <- qnorm(e.p, mean = mu, sd = sigma)
					
					# 5. draw a Q-Q plot
					plot(e.q, t.q, xlab = 'Empirical'
					, ylab = 'Theoretical', main = 'Q-Q plot')
					# reference line
					points(c(0, 100), c(0, 100), type = 'l', col = 'blue') 
				\end{verbatim}
		\end{itemize}
\end{enumerate}