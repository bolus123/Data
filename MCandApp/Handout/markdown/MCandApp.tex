% !TeX program = pdfLaTeX
\documentclass[12pt]{article}
\usepackage{amsmath}
\usepackage{graphicx,psfrag,epsf}
\usepackage{enumerate}
\usepackage{natbib}
\usepackage{url} % not crucial - just used below for the URL

%\pdfminorversion=4
% NOTE: To produce blinded version, replace "0" with "1" below.
\newcommand{\blind}{}

% DON'T change margins - should be 1 inch all around.
\addtolength{\oddsidemargin}{-.5in}%
\addtolength{\evensidemargin}{-.5in}%
\addtolength{\textwidth}{1in}%
\addtolength{\textheight}{1.3in}%
\addtolength{\topmargin}{-.8in}%

\begin{document}

\def\spacingset#1{\renewcommand{\baselinestretch}%
{#1}\small\normalsize} \spacingset{1}


%%%%%%%%%%%%%%%%%%%%%%%%%%%%%%%%%%%%%%%%%%%%%%%%%%%%%%%%%%%%%%%%%%%%%%%%%%%%%%

\if0\blind
{
  \title{\bf R handout: Monte Carlo Method and Application}

  \author{
      }
  \maketitle
} \fi

\if1\blind
{
  \bigskip
  \bigskip
  \bigskip
  \begin{center}
    {\LARGE\bf R handout: Monte Carlo Method and Application}
  \end{center}
  \medskip
} \fi

\bigskip
\begin{abstract}

\end{abstract}

\noindent%
{\it Keywords:} 
\vfill

\newpage
\spacingset{1.45} % DON'T change the spacing!

\hypertarget{methodology}{%
\section{Methodology}\label{methodology}}

\hypertarget{motivation}{%
\subsubsection{1. Motivation}\label{motivation}}

In practice, we may meet some hard integrations which is really hard to
solve. For example, in one of my working papers, we have this integrand
with the domain \(0<u<1, 0<v<1\) \begin{equation*}
            CARL = (1 - \Phi(\frac{\Phi^{-1}(u)}{\sqrt{k}} + \frac{c}{c_4}\sqrt{\frac{F^{-1}_{\chi^2_{k-1}(v)}}{k-1}}) + \Phi(\frac{\Phi^{-1}(u)}{\sqrt{k}} - \frac{c}{c_4}\sqrt{\frac{F^{-1}_{\chi^2_{k-1}(v)}}{k-1}}))^{-1}
        \end{equation*} where \(k\), \(c\) and \(c_4\) are constants.
\(\Phi\) is the c.d.f. of the standard normal distribution.
\(\Phi^{-1}\) is the quantile function the standard normal distribution.
\(F^{-1}_{\chi^2_{k-1}}\) is the quantile function of \(\chi^2\) with
\(k-1\) degrees of freedom.

\par

The analytical solution probably is desperate, but we can use other
methods to obtain the numerical solution. This handout will show a way
to solve integrations numerically.

\hypertarget{example.-y-x-where-x-sim-n0-1.-find-out-ey}{%
\subsubsection{\texorpdfstring{2. Example. \(Y = |X|\) where
\(X \sim N(0, 1)\). Find out
\(E(Y)\)}{2. Example. Y = \textbar{}X\textbar{} where X \textbackslash{}sim N(0, 1). Find out E(Y)}}\label{example.-y-x-where-x-sim-n0-1.-find-out-ey}}

\hypertarget{a-method-1-direct-integration-by-hand}{%
\paragraph{(a) Method 1: Direct Integration (By
hand)}\label{a-method-1-direct-integration-by-hand}}

As we know, the p.d.f. of \(Y\) \begin{equation*}
                    f_Y(y) = 2\phi(y), y \ge 0
                \end{equation*} where \(\phi\) is the p.d.f. of the
standard normal distribution. The expectation \begin{equation*}
                    \begin{split}
                        E(Y) &= \int_{0}^{\infty} 2y\phi(y) dy \\
                        & = \frac{1}{\sqrt{2\pi}}\int_{0}^{\infty} 2y e^{-\frac{1}{2}y^2}dy \\
                        & = \frac{2}{\sqrt{2\pi}}\int_{0}^{\infty} \frac{1}{2} e^{-\frac{1}{2}y^2}dy^2 \\
                        & = \sqrt{\frac{2}{\pi}} \approx 0.7979
                    \end{split}
                \end{equation*}

\hypertarget{b-method-2-direct-integration-in-r}{%
\paragraph{(b) Method 2: Direct Integration (In
R)}\label{b-method-2-direct-integration-in-r}}

\begin{itemize}
\item
  R code

\begin{verbatim}
# define the p.d.f. of Y
Y.pdf <- function(y) 2 * dnorm(y)

# define the integrand of the expecation of Y
E.Y <- function(y) y * Y.pdf(y)

# Integrate the integrand
integrate(E.Y, lower = 0, upper = Inf)
# Result: 0.7978846
\end{verbatim}
\end{itemize}

\hypertarget{c-monte-carlo-method}{%
\paragraph{(c) Monte Carlo Method}\label{c-monte-carlo-method}}

Suppose \(X\) is a random variable with a p.d.f. \(f\) and a c.d.f.
\(F\) over a domain \(D\). Also, we have a sample
\(X_s = \{x_1, x_2, ..., x_n\}\) from \(X\) with \(n\) observations.
\begin{equation*}
            E(X) = \int_{D}xf(x)dx \approx \bar{X} = \frac{1}{n} \sum_{i = 1}^{n}x_i
            \end{equation*} Besides, because of Central Limit Theorem,
\(\bar{X}\) asymptotically follows \(N(\mu, \frac{\sigma^2}{n})\) where
\(\mu = E(X)\) is the mean of \(X\) and \(\sigma^2\) is the variance of
\(X\). So, when \(n \rightarrow \infty\),
\(\bar{X} \xrightarrow{p} E(X)\). Hence, intuitively, the result is more
precise with a larger sample size.

\hypertarget{i.-method-3-monte-carlo-method-with-transformation-1}{%
\subparagraph{i. Method 3: Monte Carlo Method with Transformation
1}\label{i.-method-3-monte-carlo-method-with-transformation-1}}

\begin{itemize}
\tightlist
\item
  Motivation
\end{itemize}

Our target distribution may be hard to be found, but the original
distribution may be not. Hence, we simulate a sample from the original
distribution and transform it to our target, and then, calculate the
expectation.

\begin{itemize}
\tightlist
\item
  The analytical part
\end{itemize}

Because \begin{equation*}
                    E(Y) = E(|X|) = \int_{-\infty}^{\infty}|x| \phi(x)dx \approx \frac{1}{n}\sum_{i = 1}^{n} |x_i|
                    \end{equation*} where \(n\) is the number of
simulation and \(X_s = \{ x_1, ..., x_n\}\) is a sample from
\(N(0, 1)\).

\begin{itemize}
\item
  R code

\begin{verbatim}
# set seed to make this process repeatable
set.seed(12345)

# set the number of simulation
n <- 1000

# simulate a sample from the standard 
# normal distribution
X.s <- rnorm(n)

# calculate the mean of the absoluate values
mean(abs(X.s))
# Result: 0.7944
\end{verbatim}
\end{itemize}

\hypertarget{ii.-method-4-monte-carlo-method-with-transformation-2}{%
\subparagraph{ii. Method 4: Monte Carlo Method with Transformation
2}\label{ii.-method-4-monte-carlo-method-with-transformation-2}}

\begin{itemize}
\tightlist
\item
  Motivation
\end{itemize}

Both of our target distribution and the original distribution may be
hard to be found, so we use other relatively simple distributions which
have same domains as the ones of our target distribution or the original
distribution, to simulate our target, and then, calculate the
expectation.

\begin{itemize}
\tightlist
\item
  The analytical part
\end{itemize}

Because \(exp(1)\) has a same domain as \(Y\), \begin{equation*}
                \begin{split}
                    E(Y) &= \int_{0}^{\infty}yf_Y(y)dy = \int_{0}^{\infty}\frac{yf_y(y)}{e^{-y}}e^{-y}dy \\
                    &
                    = E(\frac{yf_Y(y)}{e^{-y}}) \approx \frac{1}{n}\sum_{i=1}^{n}\frac{y_if_Y(y_i)}{e^{-y_i}}
                \end{split}
                \end{equation*} where \(n\) is the number of simulation
and \(Y_s = \{y_1, ..., y_n\}\) is a sample from \(exp(1)\).

\begin{itemize}
\item
  R code

\begin{verbatim}
# set seed to make this process repeatable
set.seed(12345)

# set the number of simulation
n <- 1000

# define the p.d.f. of Y
Y.pdf <- function(y) 2 * dnorm(y)

# simulate a sample from the standard 
# normal distribution
Y.s <- rexp(n)

# calculate the mean of the absoluate values
mean(Y.s * Y.pdf(Y.s) / dexp(Y.s))
# Result: 0.7729
\end{verbatim}
\end{itemize}

\hypertarget{summary}{%
\subsubsection{3. Summary}\label{summary}}

\hypertarget{a-comparison-of-these-methods}{%
\paragraph{(a) Comparison of these
methods}\label{a-comparison-of-these-methods}}

\begin{verbatim}
##         Method1 Method2 Method3 Method4
## Results  0.7979  0.7979  0.7944  0.7729
\end{verbatim}

where Method3 and Method4 have 1000 simulations.

\hypertarget{b-steps-of-monte-carlo-method}{%
\paragraph{(b) Steps of Monte Carlo
Method}\label{b-steps-of-monte-carlo-method}}

\begin{enumerate}
\def\labelenumi{\roman{enumi}.}
\tightlist
\item
  Check the domain
\item
  Simulate an appropriate sample over the domain
\item
  Calculate the mean via an appropriate transformation
\end{enumerate}

\hypertarget{practice}{%
\subsubsection{4. Practice}\label{practice}}

Suppose \(Y = X^{-1}\), where \(X \sim \chi^2(10)\). Use Monte Carlo
Method to find \(E(Y)\).

\par

Hint: You need to show the analytical part and the coding part in R,
separately. Also, you may need to use \(rchisq\) and \(help(rchisq)\) in
R

\newpage

Application

This data comes from the Vancouver Open Data Catalogue. It has 530,652
records regarding to different types of crimes between Jan 01, 2003 and
July 13, 2017. The original data set contains coordinates in UTM Zone 10
with Latitude and Longitude. \citep{CrimeInVancouver}

Suppose our target is to predict the monthly frequency of the type of
crimes ``Break and Enter Commercial'' in the future and we are not very
interested in the exact locations of crimes. The frequency during the
period between Jan 01, 2003 and July 13, 2017 can be showed as the
following plot:

where on July 2, 2003, Vancouver won the bid to host the Winter Olympic
(the vertical red dashed line). The game was hosted from February 12 to
28, 2010 (the vertical blue dashed line). So I called the period between
July 2, 2003 and February, 28, 2010 as the preparation of Winter
Olympic. Also, on June 15, 2011, there was a riot, 2011 Vancouver
Stanley Cup riot (the vertical green dashed line). It is obvious for us
to observe that the frequency has a convex shape and one of possible
reasons causing this phenomenon is the Winter Olympic. Besides, the
fraction of data after 2017 may be not trustful, because the trend of
this fraction is supposed to be increasing or maintain the same level.

\par

Intuitively, I will guess the frequency of crimes are increasing until
reaching the situation before the preparation of Winter Olympic. Hence,
I will use the fraction of the data before July 2003, as my training
data, to predict the frequency after 2017.

Suppose \(X = \{303, 254, 292, 266, 291, 306\}\) is from my training
data and this data follows a Poisson distribution with parameter
\(\lambda\). Find out the 2.5\% and 97.5\% quantile. The p.d.f. of \(X\)
\begin{equation*}
        f_X(x) = \frac{\lambda^{x}e^{-\lambda}}{x!} , x = 0, 1, 2, ...
    \end{equation*} By m.l.e, \(\hat{\lambda} = \bar{X} = 285.3333\), so
the 2.5\% and 97.5\% quantile \begin{equation*}
        \begin{cases}
            0.025 = \sum_{t = 1}^{q_{0.025}}\frac{\lambda^{t}e^{-\lambda}}{t!} \\
            0.975 = \sum_{t = 1}^{q_{0.975}}\frac{\lambda^{t}e^{-\lambda}}{t!} 
        \end{cases} \Rightarrow \begin{cases}
            q_{0.025} = 253 \\
            q_{0.975} = 319
        \end{cases}
    \end{equation*} They can be showed on the plot

where the upper horizontal dashed line is the 97.5\% quantile, 319 and
the lower horizontal dashed line is the 2.5\% quantile, 253.

Because \(X\) is a sample, \(\hat{\lambda} = \bar{X}\) shall have its
own distribution. Suppose
\(\hat{\lambda} \sim gamma(\alpha= 1349.507, \beta = 0.2115)\). Find out
the 2.5\% and 97.5\% quantile.

\par

The p.d.f. of \(X\) given \(\hat{\lambda}\) \begin{equation*}
            f_{X|\hat{\lambda}}(x) = \frac{\lambda^{x}e^{-\lambda}}{x!}
    \end{equation*}

\par

And the p.d.f. of \(\hat{\lambda}\) \begin{equation*}
        f_{\hat{\lambda}}(\lambda) = \frac{1}{\Gamma(\alpha)\beta^{\alpha}}\lambda^{\alpha - 1}e^{-\frac{\lambda}{\beta}}d\lambda, \lambda > 0
    \end{equation*} Then \begin{equation*}
        \begin{split}
            P(X \le x) &= \int_{0}^{\infty}P(X < x | \hat{\lambda} = \lambda)f_{\hat{\lambda}}(\lambda) d \lambda \\
            & = \int_{0}^{\infty}(\sum_{t = 1}^{x}\frac{\lambda^{t}e^{-\lambda}}{t!})(\frac{1}{\Gamma(\alpha)\beta^{\alpha}}\lambda^{\alpha - 1}e^{-\frac{\lambda}{\beta}}) d \lambda
        \end{split}
    \end{equation*} So, the 2.5\% and 97.5\% quantile \begin{equation*}
        \begin{cases}
        0.025 = \int_{0}^{\infty}(\sum_{t = 1}^{q_{0.025}}\frac{\lambda^{t}e^{-\lambda}}{t!})(\frac{1}{\Gamma(\alpha)\beta^{\alpha}}\lambda^{\alpha - 1}e^{-\frac{\lambda}{\beta}}) d \lambda \\
        0.975 = \int_{0}^{\infty}(\sum_{t = 1}^{q_{0.975}}\frac{\lambda^{t}e^{-\lambda}}{t!})(\frac{1}{\Gamma(\alpha)\beta^{\alpha}}\lambda^{\alpha - 1}e^{-\frac{\lambda}{\beta}}) d \lambda
        \end{cases}
    \end{equation*} The analytical solution is complicated, but it is
relatively easy if we use numerical methods. By Monte Carlo Method,
\begin{equation*}
        P(X \le x) = \int_{0}^{\infty}(\sum_{t = 1}^{x}\frac{\lambda^{t}e^{-\lambda}}{t!})(\frac{1}{\Gamma(\alpha)\beta^{\alpha}}\lambda^{\alpha - 1}e^{-\frac{\lambda}{\beta}}) d \lambda = E(\sum_{t = 1}^{x}\frac{\lambda^{t}e^{-\lambda}}{t!})) \approxeq \frac{1}{n} \sum_{i = 1}^{n} \sum_{t = 1}^{x}\frac{\lambda_i^{t}e^{-\lambda_i}}{t!}
    \end{equation*} where \(\lambda_s = \{\lambda_1,...,\lambda_n \}\)
is a sample following \(gamma(\alpha= 1349.507, \beta = 0.2115)\). Also,
the equation can be expressed as the following (the root searching form)

\begin{equation*}
        P(X \le x) - \frac{1}{n} \sum_{i = 1}^{n} \sum_{t = 1}^{x}\frac{\lambda_i^{t}e^{-\lambda_i}}{t!} = 0
    \end{equation*}

where \(P(X \le q_{0.025}) = 0.025\) and \(P(X \le q_{0.975}) = 0.975\).
The quantiles \(q_{0.025}\) and \(q_{0.975}\) are roots of the equation.
After the computation, we can show the quantiles on the plot:

where the upper horizontal solid line is the 97.5\% quantile, 323, and
the lower horizontal solid line is the 2.5\% quantile, 250.

\begin{itemize}
\item
  R code for the whole process

\begin{verbatim}
#################################################
# describe BEC during the period
#################################################
# load the new data
addr <- paste('https://raw.githubusercontent.com/bolus123', 
'/R-handout/master/MCandApp/BEC.csv', sep = '')
BEC.monthly.freq <- read.csv(file = addr)[, -1]

# add a new column combining YEAR with MONTH
BEC.monthly.freq <- cbind(BEC.monthly.freq, 
paste(BEC.monthly.freq$YEAR, 
substr(
as.character(
as.numeric(
BEC.monthly.freq$MONTH) + 
100), 2, 3)
, sep = '-'))

# build a basic scatter frequency plot 
# during the period
plot(BEC.monthly.freq[, 5], BEC.monthly.freq[, 4], 
xaxt="n", ylab = 'Frequency')
# define the tick of x-axis
labs <- sort(BEC.monthly.freq[, 5])[rep(c(T, F, F, 
F, F, F, F, F, F, F, F, F), 15)]
for (i in 1:15){

axis(1, at = (12 * (i - 1) + 1), 
labels = labs[i], las = 2)

}

# specify special months
prepare.Winter.Olympic <- BEC.monthly.freq[, 5] == '2003-06'
Winter.Olympic <- BEC.monthly.freq[, 5] == '2010-02'
Stanley.Cup.riot <- BEC.monthly.freq[, 5] == '2011-06'

# show Preparation of Winter Olympic on the plot
abline(v = BEC.monthly.freq[prepare.Winter.Olympic, 5], 
col = 'red', lty = 2)
text(BEC.monthly.freq[prepare.Winter.Olympic, 5], 5, 
'Preparation of Winter Olympic', pos = 4, 
srt = 90, cex = 0.8)

# show Winter Olympic on the plot 
abline(v = BEC.monthly.freq[Winter.Olympic, 5], 
col = 'blue', lty = 2)
text(BEC.monthly.freq[Winter.Olympic, 5], 5, 
'Winter Olympic', pos = 4, srt = 90, cex = 0.8)

# show Stanley Cup Riot on the plot 
abline(v = BEC.monthly.freq[Stanley.Cup.riot, 5], 
col = 'green', lty = 2)
text(BEC.monthly.freq[Stanley.Cup.riot, 5], 5, 
'Stanley Cup Riot', pos = 4, srt = 90, cex = 0.8)

# set the maximun date of the training data
x1.max.date <- which(
BEC.monthly.freq[
order(BEC.monthly.freq[, 5]), 5] == '2003-06')

# cut it off from the original data
x1 <- BEC.monthly.freq[
order(BEC.monthly.freq[, 5]), 4][1:x1.max.date]

# fit a poisson model for the training data
lambda1 <- mean(x1)

# show the 2.5% and 97.5 quantiles on the plot
abline(h = qpois(0.975, lambda1), lty = 2)
abline(h = qpois(0.025, lambda1), lty = 2)

#################################################
# find the gamma distribution for lambda 
# by the nonparametric bootstrap
#################################################
set.seed(12345)

# the number of times for bootstrapping
n <- 100000
# set a vector to carry the means
xs.means <- rep(NA, n)

for (i in 1:n){
# resample from the training data
xs <- sample(x1, 6, replace = T)
# calculate their means
xs.means[i] <- mean(xs)
}

# calculate the grand mean
mu <- mean(xs.means)
# calculate the grand variance
sigma2 <- var(xs.means)

# fit a gamma distribution by the method of moment
alpha <- mu^2 / sigma2
beta <- sigma2 / mu

# check the gamma distribution
#x <- 1:1000 
#plot(x, dgamma(x, alpha, 1/beta), type = 'l')

#################################################
# fit a new poisson distribution 
# with a parameter lambda 
# which is a random variable
#################################################
# set a seed make this process repeatable
set.seed(12345)

# define a user-defined function
# p is P(X <= x)
# alpha is the alpha for gamma(alpha, beta)
# beta is the beta for gamma(alpha, beta)
# interval is the range of searching the quantile
# rnum is the number of simulations
X.quantile <- function(p, alpha, beta, 
interval = c(100, 500), rnum = 10000){

root.finding <- function(x, p, lambda){
# The root searching form
p - mean(ppois(x, lambda))
}

# simulate a sample from gamma distribution
lambda <- rgamma(rnum, alpha, scale = beta)

# search the root by the bisection method
uniroot(root.finding, interval = interval, 
p = p, lambda = lambda)$root

}

# calcualte the 2.5% quantile
q0025 <- X.quantile(p = 0.025, 
alpha = alpha, beta = beta)
# calcualte the 97.5% quantile
q0975 <- X.quantile(p = 0.975, alpha = alpha, beta = beta)

# add horizontal lines on the plot
abline(h = q0025)
abline(h = q0975)
\end{verbatim}
\end{itemize}

\bibliographystyle{agsm}
\bibliography{C:/Github/R-handout/MCandApp/Handout/markdown/master.bib}

\end{document}
