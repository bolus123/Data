\begin{enumerate}
	\item Monte Carlo Method
	\par Suppose $X$ is a random variable with a p.d.f. $f$ and a c.d.f. $F$ over a domain $D$. Also, we have a sample $X_s = \{x_1, x_2, ..., x_n\}$ from $X$ with $n$ observations.
	\begin{equation*}
		E(X) = \int_{D}xf(x)dx \approx \bar{X} = \frac{1}{n} \sum_{i = 1}^{n}x_i
	\end{equation*}
	Besides, because of Central Limit Theorem, $\bar{X}$ asymptotically follows $N(\mu, \frac{\sigma^2}{n})$ where $\mu = E(X)$ is the mean of $X$ and $\sigma^2$ is the variance of $X$. So, when $n \rightarrow \infty$, $\bar{X} \rightarrow E(X)$. Hence, intuitively, the result is more precise with a larger sample size.
	\item Example. $Y = |X|$ where $X \sim N(0, 1)$. Find out $E(Y)$
		\begin{enumerate}
			\item Direct Integration
				\par As we know, the p.d.f. of $Y$
				\begin{equation*}
					f_Y(y) = 2\phi(y), y \ge 0
				\end{equation*}
				where $\phi$ is the p.d.f. of the standard normal distribution.
				\begin{itemize}
					\item R code
					\begin{verbatim}
						# define the p.d.f. of Y
						Y.pdf <- function(y) 2 * dnorm(y)
						
						# define the integrand of the expecation of Y
						E.Y <- function(y) y * Y.pdf(y)
						
						# Integrate the integrand
						integrate(E.Y, lower = 0, upper = Inf)
						# Result: 0.7978846
					\end{verbatim}
				\end{itemize}
			\item Monte Carlo Method with a transformation
				\par Because 
				\begin{equation*}
					E(Y) = E(|X|) = \int_{-\infty}^{\infty}|x| \phi(x)dx \approx \frac{1}{n}\sum_{i = 1}^{n} |x_i|
				\end{equation*}
				where $n$ is the number of simulation and $X_s = \{ x_1, ..., x_n\}$ is a sample from $N(0, 1)$.
				\begin{itemize}
					\item R code
					\begin{verbatim}
						# set seed to make this process repeatable
						set.seed(12345)
						
						# set the number of simulation
						n <- 1000
						
						# simulate a sample from the standard normal distribution
						X.s <- rnorm(n)
						
						# calculate the mean of the absoluate values
						mean(abs(X.s))
						# Result: 0.7944
					\end{verbatim}
				\end{itemize}
			\item Monte Carlo Method with another transformation
				\par Because $exp(1)$ has a same domain as $Y$, 
				\begin{equation*}
					E(Y) = \int_{0}^{\infty}yf_Y(y)dy = \int_{0}^{\infty}\frac{yf_y(y)}{e^{-y}}e^{-y}dy = E(\frac{yf_Y(y)}{e^{-y}}) \approx \frac{1}{n}\sum_{i=1}^{n}\frac{y_if_Y(y_i)}{e^{-y_i}}
				\end{equation*}
				where $n$ is the number of simulation and $Y_s = \{y_1, ..., y_n\}$ is a sample from $exp(1)$.
				\begin{itemize}
					\item R code
					\begin{verbatim}
						# set seed to make this process repeatable
						set.seed(12345)
						
						# set the number of simulation
						n <- 1000
						
						# define the p.d.f. of Y
						Y.pdf <- function(y) 2 * dnorm(y)
						
						# simulate a sample from the standard normal distribution
						Y.s <- rexp(n)
						
						# calculate the mean of the absoluate values
						mean(Y.s * Y.pdf(Y.s) / dexp(Y.s))
						# Result: 0.7729
					\end{verbatim}
				\end{itemize}
		\end{enumerate}
	\item Summary: Steps of Monte Carlo Method
		\begin{enumerate}
			\item Check the domain
			\item Simulate an appropriate sample over the domain 
			\item Calculate the mean via an appropriate transformation
		\end{enumerate}
	\item Practice: suppose $Y = X^{-1}$, where $X \sim \chi^2(10)$. Use Monte Carlo Method to find $E(Y)$.
	\par Hint: You need to show the analytical part and the numerical part in R. Also, you may need to use $rchisq$ and $help(rchisq)$ in R
\end{enumerate}